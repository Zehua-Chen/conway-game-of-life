\documentclass{article}
\usepackage[margin=1in]{geometry}
\usepackage{inconsolata}
\usepackage{dirtytalk}
\usepackage{tikz}
\usepackage{float}
\usepackage{listings}
\usepackage{hyperref}

\usetikzlibrary{automata}

\title{COMS 4995 Final Project}
\author{Zehua Chen (zc2616)}
\date{\today}

\lstset{
  basicstyle=\footnotesize\ttfamily,
  numbers=left
}

\newcommand{\code}[1]{{\ttfamily #1}}

\begin{document}
  \maketitle
  \tableofcontents

  \setlength{\parskip}{8px}
  \setlength{\parindent}{0px}

  \section{Implementation}

    \subsection{Coordinate System}

      The world has a height and a width. The origin (0, 0) is located at the
      center of the world. If we save the world as an image, positive
      axis would be on the right of the image and positive y axis
      would be on the top of the image.

      In order for a cell to exist at (0, 0) and the world be modeled using
      width and height, width and height must be odd numbers

    \subsection{Single Core}

      Single core implementation (\code{Conway.Simulate.simulateSync})
      of the simulation:

      \begin{enumerate}
        \item \label{single: step-simulate} Simulate the world
        \item \label{single: step-grow} Simulate one layer outside of the existing world, and expand
        the existing world if needed
      \end{enumerate}

    \subsection{Multi Core}

      Multicore implementation (\code{Conway.Simulate.simulateAsync})
      of the simulation

      \begin{enumerate}
        \item \label{multi: step-partition}
        Divide the world into partitions and calculate adjacent cells
        between partitions
        \item \label{multi: step-simulate}
        Simulate the partitions, with access only to the world within the
        partition. \emph{This step reuses code of step \ref{single: step-simulate} from
        single core implementation}
        \item \label{multi: step-simulate-border}
        Simulate the adjacent cells between partitions, with access to the
        whole world.
        \item \label{multi: step-grow}
        Simulate one layer outside of the existing world. \emph{This step reuses
        code of step \ref{single: step-simulate} from single core implementation}
        \item Combine the result of \ref{multi: step-simulate},
        \ref{multi: step-simulate-border} and \ref{multi: step-grow}
      \end{enumerate}

      Of the above steps, the operations are \ref{multi: step-partition}
      is performed in parallel first. After they finish, the operations in
      \ref{multi: step-simulate}, \ref{multi: step-simulate-border},
      \ref{multi: step-grow} are performed in parallel. The latter operations
      are performed in a second step because they depend on the result from
      the former operations.

      \subsubsection{Parameters}

        Multicore simulation allows customizations in

        \begin{itemize}
          \item slice width, slice height: how big each slice should be
          \item chunk size: the multi core simulation uses \code{parList} in
          some operations. Chunk size is used to customize \code{parList}
        \end{itemize}

    \subsection{Modules}

      \begin{enumerate}
        \item \code{Conway.World}: provide abstraction to the conway world,
        and various utility functions useful for operations on the world
        \item \code{Conway.Slice}: provide abstraction of slices of worlds
        \item \code{Conway.Partition}: provide function that split the world
        into slices, and find the cells on the outer layer of each partition
        \item \code{Conway.Simulate}: implements single and multi core
        simulation
        \item \code{Conway.PPM}: saves a conway world into a PPM image
      \end{enumerate}

    \subsection{Testing}

      \begin{enumerate}
        \item \code{test/World.hs}: tests for \code{Conway.World}
        \item \code{test/Partition/Partition.hs}: make sure the world can be
        divided into partitions
        \item \code{test/Partition/PartitionBorders.hs}: make sure the outer
        most layer of a partition can be resolved property
        \item \code{test/Simulate/Finite.hs}: make sure that simulation on
        finite grid is correct
        \item \code{test/Simulate/Grow.hs}: make sure that teh grid is grown
        when needed
        \item \code{test/Simulate/Infinite.hs}: make sure that single core
        and multi core implementation produces the same result from simulating
        on an infinite grid.
      \end{enumerate}

  \section{Performance}
  \section{Code}

    \subsection{Implementation}

      \lstinputlisting[caption=\code{Conway.Partition}]{../src/Conway/Partition.hs}
      \lstinputlisting[caption=\code{Conway.PPM}]{../src/Conway/PPM.hs}
      \lstinputlisting[caption=\code{Conway.Simulate}]{../src/Conway/Simulate.hs}
      \lstinputlisting[caption=\code{Conway.Slice}]{../src/Conway/Slice.hs}
      \lstinputlisting[caption=\code{Conway.World}]{../src/Conway/World.hs}

    \subsection{Testing}

      \lstinputlisting[caption=\code{test/World.hs}]{../test/World.hs}
      \lstinputlisting[caption=\code{test/Partition/Partition.hs}]{../test/Partition/Partition.hs}
      \lstinputlisting[caption=\code{test/Partition/PartitionBorder.hs}]{../test/Partition/PartitionBorder.hs}
      \lstinputlisting[caption=\code{test/Simulate/Finite.hs}]{../test/Simulate/Finite.hs}
      \lstinputlisting[caption=\code{test/Simulate/Grow.hs}]{../test/Simulate/Grow.hs}
      \lstinputlisting[caption=\code{test/Simulate/Infinite.hs}]{../test/Simulate/Infinite.hs}

\end{document}
