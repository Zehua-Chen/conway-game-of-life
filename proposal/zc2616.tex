\documentclass{article}
\usepackage{listings}
\usepackage{inconsolata}
\usepackage{hyperref}

\title{COMS 4995 Functional Parallel Programming - Conway Game of Life}
\author{Zehua Chen (zc2616)}

\setlength{\parindent}{0pt}

\lstset{
  basicstyle=\ttfamily\footnotesize,
  numbers=left,
}

\begin{document}
  \maketitle

  \section{Overview}

    The goal of the project is to build a conway game of life simulator for
    an infinite grid for a specified number of iterations. The simulator would
    take a list of living cells as the input and save iterations, which are
    also made up of lists of living cells, to disk.

  \section{IO}

    \lstinputlisting{../engine/Schema.hs}

    \begin{itemize}
      \item The input of a story will just be a story of one page
      \item The data will be serialized and deserialized as JSON using
      \href{https://hackage.haskell.org/package/aeson}{aeson}
      \item The number of iterations wille be passed in from the command line
    \end{itemize}

  \subsection{Implementation}

    The conway grid will be decomposed into a subgrids for parallelization, with
    additional step to solve interactions between subgrids and resize the overall grid
    if needed

  \section{References}

    Below are some websites that I think that will come in handy for implementing
    the project.

    \begin{itemize}
      \item \href{https://en.wikipedia.org/wiki/Conway%27s_Game_of_Life}{Wikipedia}:
      rules of Conway game of life
      \item \href{https://www.madelyneriksen.com/python-game-of-life}{Conway's Game of Life With Pure Python}
      contains a pure \emph{python version of conway game of life using sparse rather}
      than dense input.
    \end{itemize}

\end{document}
