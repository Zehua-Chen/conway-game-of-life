\documentclass{article}
\usepackage{listings}
\usepackage{inconsolata}
\usepackage{hyperref}

\title{COMS 4995 Functional Parallel Programming - Conway Game of Life}
\author{Zehua Chen (zc2616)}

\setlength{\parindent}{0pt}

\lstset{
  basicstyle=\ttfamily\footnotesize,
  numbers=left,
}

\begin{document}
  \maketitle
  \tableofcontents

  \section{Overview}

    The project is broken down into two parts

    \begin{itemize}
      \item An \textbf{engine written in Haskell} that takes a input file and simulate
      conway game of life for a number of iterations and save the iterations
      to an output file
      \item A \textbf{simple web interface} helps create the input file and takes the
      output file and replay it
    \end{itemize}

  \section{Communication}

    Communication between the web interface and the Haskell engine will be done
    using JSON

    Input and output files will be of the same format. The only difference
    is that the input file would just contain one iteration

    \lstinputlisting{../frontend/schema.ts}

  \section{Engine}

    \subsection{Dependencies}

      \begin{itemize}
        \item \href{https://hackage.haskell.org/package/aeson-2.0.2.0/docs/Data-Aeson.html}{aeson}:
        used for JSON parsing
        \item \href{https://hackage.haskell.org/package/argparser-0.3.4/docs/System-Console-ArgParser.html}{argparse}:
        command line argument parser
        \item \href{https://hackage.haskell.org/package/HUnit-1.6.2.0/docs/Test-HUnit.html}{HUnit}:
        used for unit testing
      \end{itemize}

  \section{Web Interface}

    The web interface will be hosted using Github Pages

    \subsection{Dependencies}

      \begin{itemize}
        \item \href{https://www.gatsbyjs.com}{Gatsby}: static site generation
        for React.js
      \end{itemize}
\end{document}
